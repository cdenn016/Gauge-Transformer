\documentclass[11pt,a4paper]{article}

% ============================================================================
% PACKAGES
% ============================================================================
\usepackage[utf8]{inputenc}
\usepackage[T1]{fontenc}
\usepackage{amsmath,amssymb,amsthm}
\usepackage{mathtools}
\usepackage{bm}
\usepackage{bbm}
\usepackage{geometry}
\usepackage{hyperref}
\usepackage{cleveref}
\usepackage{booktabs}
\usepackage{enumitem}
\usepackage{tcolorbox}
\usepackage{xcolor}

\geometry{margin=1in}

% ============================================================================
% THEOREM ENVIRONMENTS
% ============================================================================
\theoremstyle{plain}
\newtheorem{theorem}{Theorem}[section]
\newtheorem{lemma}[theorem]{Lemma}
\newtheorem{proposition}[theorem]{Proposition}
\newtheorem{corollary}[theorem]{Corollary}

\theoremstyle{definition}
\newtheorem{definition}[theorem]{Definition}
\newtheorem{axiom}{Axiom}
\newtheorem{example}[theorem]{Example}

\theoremstyle{remark}
\newtheorem{remark}[theorem]{Remark}

% ============================================================================
% CUSTOM COMMANDS
% ============================================================================
\newcommand{\KL}{\mathrm{KL}}
\newcommand{\E}{\mathbb{E}}
\newcommand{\R}{\mathbb{R}}
\newcommand{\N}{\mathcal{N}}
\newcommand{\M}{\mathcal{M}}
\newcommand{\F}{\mathcal{F}}
\newcommand{\B}{\mathcal{B}}
\newcommand{\tr}{\mathrm{tr}}
\newcommand{\diag}{\mathrm{diag}}
\newcommand{\softmax}{\mathrm{softmax}}
\newcommand{\dd}{\mathrm{d}}
\newcommand{\pp}{\partial}
\newcommand{\T}{\mathsf{T}}

% ============================================================================
% TITLE
% ============================================================================
\title{%
    \textbf{From Gauge-Theoretic Variational Inference to Continuous Thought Machines:}\\[0.5em]
    \large A Derivation from First Principles
}

\author{
    Gauge-Transformer Project\\
    \texttt{github.com/cdenn016/Gauge-Transformer}
}

\date{January 2026}

% ============================================================================
% DOCUMENT
% ============================================================================
\begin{document}

\maketitle

\begin{abstract}
We establish a mathematical correspondence between the gauge-theoretic variational inference framework and the Continuous Thought Machine (CTM) architecture. The gauge-theoretic framework derives its variational free energy from a normalized generative model with auxiliary agreement variables, yielding inter-agent coupling terms that are necessarily of forward Kullback-Leibler form under exponential family closure. We demonstrate that the structural elements of CTM---neuron-level models, synchronization representations, and adaptive computation---admit interpretation within this framework. The correspondence is mediated by the identity relating mutual information to KL divergence from the joint distribution to the product of marginals. We distinguish carefully between the Fisher-Rao metric, which defines proper time as information-theoretic arc length, and the Hessian of the free energy, which defines effective dynamical mass incorporating constraints from priors, observations, and social coupling. The framework extends naturally to exponential families beyond Gaussians, suggesting that the mathematical structure underlying CTM is not specific to any particular distributional assumption.
\end{abstract}

\tableofcontents
\newpage

% ============================================================================
\section{Introduction}
\label{sec:introduction}
% ============================================================================

The Continuous Thought Machine (CTM) introduced by Darlow et al.\ represents an architectural departure from standard transformers through three principal innovations: neuron-level models wherein each neuron maintains private weights processing its activation history; synchronization representations wherein pairwise temporal correlations between neurons constitute the latent space; and adaptive computation wherein internal iterations proceed independently of input length, terminating upon satisfaction of confidence criteria.

Separately, gauge-theoretic approaches to multi-agent inference propose that agents maintaining probabilistic beliefs over latent states can be understood as sections of fiber bundles over a base manifold, with inter-agent communication mediated by gauge transport operators that account for differences in local reference frames. The natural coupling between such agents involves the Kullback-Leibler divergence between an agent's belief and the gauge-transported belief of its neighbor.

This paper investigates whether and to what extent these two frameworks are mathematically related. The investigation proceeds as follows. Section~\ref{sec:generative_model} presents the derivation of the gauge-theoretic variational free energy from a normalized generative model. Section~\ref{sec:uniqueness} establishes the conditional uniqueness of the forward KL divergence as the coupling term under exponential family constraints. Section~\ref{sec:exponential_families} extends the framework beyond Gaussians. Section~\ref{sec:proper_time} defines proper time as information-theoretic arc length. Section~\ref{sec:mass_matrix} distinguishes the Fisher-Rao metric from the dynamical mass matrix. Section~\ref{sec:ctm_correspondence} develops the correspondence with CTM. Section~\ref{sec:gaps} identifies assumptions and limitations.

% ============================================================================
\section{The Generative Model and Variational Free Energy}
\label{sec:generative_model}
% ============================================================================

We derive the variational free energy from a normalized generative model rather than postulating it directly. This derivation, following the framework developed in the companion manuscript, justifies the specific form of inter-agent coupling as a consequence of consistency constraints mediated by auxiliary agreement variables.

\subsection{Latent Variables and Statistical Manifold Structure}

Consider a collection of agents indexed by $i \in \{1, \ldots, N\}$. Each agent maintains two latent variables:
\begin{align}
k_i &\in \R^{d_q} \quad \text{(belief latent)}, \\
m_i &\in \R^{d_p} \quad \text{(model latent)}.
\end{align}

The belief latent $k_i$ represents agent $i$'s estimate of the current world state. The model latent $m_i$ represents agent $i$'s estimate of the parameters governing state transitions and observations. This separation reflects a hierarchical Bayesian structure: uncertainty about states given a model, and uncertainty about the model itself.

Each agent's epistemic state is characterized by Gaussian distributions over these latents:
\begin{align}
q_i(k_i) &= \N(k_i; \mu_{q,i}, \Sigma_{q,i}), \\
s_i(m_i) &= \N(m_i; \mu_{p,i}, \Sigma_{p,i}),
\end{align}
where $\mu_{q,i} \in \R^{d_q}$, $\Sigma_{q,i} \in \R^{d_q \times d_q}$ with $\Sigma_{q,i} \succ 0$, and similarly for the model parameters.

The space of such Gaussian distributions forms a statistical manifold. Let $\B_q$ denote the manifold of $d_q$-dimensional Gaussians and $\B_p$ the manifold of $d_p$-dimensional Gaussians. The full epistemic state of each agent lies in the product manifold $\B = \B_q \times \B_p$.

\subsection{Base Priors}

Each agent possesses independent Gaussian priors encoding agent-specific inductive biases:
\begin{align}
p_i(k_i) &= \N(k_i; \mu_{0,i}^{(q)}, \Sigma_{0,i}^{(q)}), \\
r_i(m_i) &= \N(m_i; \mu_{0,i}^{(p)}, \Sigma_{0,i}^{(p)}).
\end{align}

These priors are local to each agent's reference frame and need not be comparable across agents until transported via gauge operators.

\subsection{Gauge Transport Operators}

When agents occupy different locations or possess different reference frames, comparing their beliefs requires a transport operation. Let $G$ be a Lie group (we consider $G = SO(N)$ for concreteness) and let $\phi_i: \mathcal{U}_i \to \mathfrak{g}$ be agent $i$'s gauge frame field, where $\mathfrak{g} = \text{Lie}(G)$ is the Lie algebra.

The gauge transport operator from agent $j$'s frame to agent $i$'s frame is:
\begin{equation}
\Omega_{ij} = \exp(\phi_i) \exp(-\phi_j) \in G.
\end{equation}

This operator acts on Gaussian distributions via the group representation $\rho: G \to \text{GL}(d)$:
\begin{equation}
\Omega_{ij} \cdot q_j := \N(\rho(\Omega_{ij})\mu_{q,j}, \rho(\Omega_{ij})\Sigma_{q,j}\rho(\Omega_{ij})^\T).
\end{equation}

For orthogonal representations, $\rho(\Omega_{ij})^\T = \rho(\Omega_{ij})^{-1}$, preserving the positive-definiteness of covariance matrices.

\subsection{Auxiliary Agreement Variables}

To construct a normalized generative model that induces pairwise coupling between agents, we introduce auxiliary agreement variables for each ordered pair $(i, j)$:
\begin{align}
z_{ij} &\in \R^{d_q} \quad \text{(belief agreement)}, \\
w_{ij} &\in \R^{d_p} \quad \text{(model agreement)}.
\end{align}

The agreement variable $z_{ij}$ represents ``what agent $i$ believes agent $j$'s belief looks like after transport into agent $i$'s frame.'' These variables serve as latent mediators: after marginalization, they induce effective pairwise potentials between agents without requiring unnormalized Markov random field constructions.

The key construction is that $z_{ij}$ is drawn from the product of two Gaussians, each with precision $\Lambda_{ij}$:
\begin{equation}
p(z_{ij} \mid k_i, k_j) \propto \N(z_{ij}; k_i, \Lambda_{ij}^{-1}) \cdot \N(z_{ij}; \Omega_{ij} k_j, \Lambda_{ij}^{-1}).
\label{eq:agreement_coupling}
\end{equation}

The first factor anchors $z_{ij}$ to agent $i$'s own latent $k_i$. The second factor anchors $z_{ij}$ to agent $j$'s latent after transport. The precision $\Lambda_{ij}$ controls the strength of coupling: large $\Lambda_{ij}$ forces $z_{ij}$ to lie close to both $k_i$ and $\Omega_{ij} k_j$, which is possible only when these two quantities are themselves close.

\subsection{The Normalized Joint Distribution}

The complete generative model is:
\begin{equation}
\boxed{
\begin{aligned}
p&(\{k_i\}, \{m_i\}, \{z_{ij}\}, \{w_{ij}\}) \\
&= \left[\prod_i p_i(k_i) r_i(m_i)\right] \\
&\quad \times \left[\prod_{i,j} \N(z_{ij}; k_i, \Lambda_{ij}^{-1}) \N(z_{ij}; \Omega_{ij} k_j, \Lambda_{ij}^{-1})\right] \\
&\quad \times \left[\prod_{i,j} \N(w_{ij}; m_i, \Gamma_{ij}^{-1}) \N(w_{ij}; \tilde{\Omega}_{ij} m_j, \Gamma_{ij}^{-1})\right].
\end{aligned}
}
\label{eq:joint_generative}
\end{equation}

\begin{proposition}[Normalization]
The joint distribution \eqref{eq:joint_generative} is properly normalized.
\end{proposition}

\begin{proof}
Each base prior $p_i(k_i)$ and $r_i(m_i)$ is a normalized Gaussian by construction. For the agreement factors, observe that the product of two Gaussians with identical precision $\Lambda$ centered at $a$ and $b$ respectively is:
\begin{equation}
\N(z; a, \Lambda^{-1}) \N(z; b, \Lambda^{-1}) \propto \N\left(z; \frac{a+b}{2}, (2\Lambda)^{-1}\right) \cdot \exp\left(-\frac{\Lambda}{4}\|a - b\|^2\right).
\end{equation}
The first factor is a normalized Gaussian in $z$ (integrating to 1 over $z$). The second factor depends only on $a$ and $b$, not on $z$. Thus integration over $z_{ij}$ yields a function of $k_i$ and $k_j$ alone, and the overall integral over all variables factorizes into products of normalizable terms.

This stands in contrast to unnormalized Markov random fields of the form $p(\{k_i\}) \propto \exp[-\sum_{ij} \psi_{ij}(k_i, k_j)]$, where the partition function is generally intractable. The agreement variable construction guarantees $Z = 1$ by design.
\end{proof}

\subsection{Derivation of the Variational Free Energy}

We adopt a mean-field variational approximation:
\begin{equation}
q(\{k_i\}, \{m_i\}) = \prod_i q_i(k_i) s_i(m_i),
\end{equation}
with Gaussian factors as specified above.

The variational free energy is:
\begin{equation}
\F = \E_q[\log q(\{k_i\}, \{m_i\})] - \E_q[\log p(\{k_i\}, \{m_i\})] - \E_q[\log p(o \mid \{k_i\}, \{m_i\})],
\end{equation}
where the marginal prior $p(\{k_i\}, \{m_i\})$ is obtained by integrating out the agreement variables from \eqref{eq:joint_generative}, and $p(o \mid \cdot)$ is the observation likelihood.

Expanding the expectations and performing the integration over agreement variables yields quadratic forms. Specifically, the contribution from the belief agreement terms is:
\begin{equation}
\frac{1}{4} \sum_{i,j} \E_{q_i q_j}\left[(k_i - \Omega_{ij} k_j)^\T \Lambda_{ij} (k_i - \Omega_{ij} k_j)\right].
\end{equation}

For independent Gaussians $q_i = \N(\mu_{q,i}, \Sigma_{q,i})$ and $q_j = \N(\mu_{q,j}, \Sigma_{q,j})$, this expectation evaluates to:
\begin{equation}
\E_{q_i q_j}\left[(k_i - \Omega_{ij} k_j)^\T \Lambda_{ij} (k_i - \Omega_{ij} k_j)\right] = \tr(\Lambda_{ij} \Sigma_\delta) + \bar{\delta}^\T \Lambda_{ij} \bar{\delta},
\end{equation}
where $\bar{\delta} = \mu_{q,i} - \Omega_{ij} \mu_{q,j}$ and $\Sigma_\delta = \Sigma_{q,i} + \Omega_{ij} \Sigma_{q,j} \Omega_{ij}^\T$.

\subsection{Relating Quadratic Forms to KL Divergences}

The KL divergence between Gaussians $q_i = \N(\mu_i, \Sigma_i)$ and $\Omega_{ij} q_j = \N(\Omega_{ij} \mu_j, \Omega_{ij} \Sigma_j \Omega_{ij}^\T)$ is:
\begin{equation}
\begin{aligned}
D_{\KL}(q_i \| \Omega_{ij} q_j) &= \frac{1}{2}\Big[\log \frac{|\Omega_{ij} \Sigma_j \Omega_{ij}^\T|}{|\Sigma_i|}  + \tr\left((\Omega_{ij} \Sigma_j \Omega_{ij}^\T)^{-1} \Sigma_i\right) \\
&\quad + (\mu_i - \Omega_{ij} \mu_j)^\T (\Omega_{ij} \Sigma_j \Omega_{ij}^\T)^{-1} (\mu_i - \Omega_{ij} \mu_j) - d_q\Big].
\end{aligned}
\end{equation}

We now make a specific choice for the coupling precision:
\begin{equation}
\Lambda_{ij} := \tau_{ij}^{(q)} (\Omega_{ij} \Sigma_{q,j} \Omega_{ij}^\T)^{-1},
\label{eq:precision_choice}
\end{equation}
where $\tau_{ij}^{(q)} > 0$ is a dimensionless coupling strength.

In the alignment regime where $\Sigma_{q,i} \approx \Omega_{ij} \Sigma_{q,j} \Omega_{ij}^\T$ (i.e., transported covariances approximately match), the log-determinant and trace terms approximately cancel, and the quadratic expectation becomes:
\begin{equation}
\frac{1}{4} \E_{q_i q_j}\left[(k_i - \Omega_{ij} k_j)^\T \Lambda_{ij} (k_i - \Omega_{ij} k_j)\right] \approx \frac{\tau_{ij}^{(q)}}{2} D_{\KL}(q_i \| \Omega_{ij} q_j) + C,
\end{equation}
where $C$ absorbs dimension-dependent constants and $O(\|\Delta\Sigma\|^2)$ corrections.

Defining normalized alignment weights $\beta_{ij} := \tau_{ij}^{(q)} / 2$ and $\gamma_{ij} := \tau_{ij}^{(p)} / 2$, and collecting all terms, we obtain the final form of the variational free energy:

\begin{tcolorbox}[colback=blue!5,colframe=blue!50,title=Variational Free Energy]
\begin{equation}
\boxed{
\begin{aligned}
\F[\{q_i\}, \{s_i\}] &= \sum_i D_{\KL}(q_i \| p_i) + \sum_i D_{\KL}(s_i \| r_i) \\
&\quad + \sum_{i,j} \beta_{ij} D_{\KL}(q_i \| \Omega_{ij} q_j) \\
&\quad + \sum_{i,j} \gamma_{ij} D_{\KL}(s_i \| \tilde{\Omega}_{ij} s_j) \\
&\quad - \E_q[\log p(o \mid \{k_i\}, \{m_i\})].
\end{aligned}
}
\label{eq:vfe_final}
\end{equation}
\end{tcolorbox}

The five terms have distinct interpretations:
\begin{enumerate}
\item $D_{\KL}(q_i \| p_i)$: Belief complexity cost, regularizing toward prior expectations.
\item $D_{\KL}(s_i \| r_i)$: Model complexity cost, regularizing toward hyperprior.
\item $\beta_{ij} D_{\KL}(q_i \| \Omega_{ij} q_j)$: Belief alignment, enforcing epistemic consensus.
\item $\gamma_{ij} D_{\KL}(s_i \| \tilde{\Omega}_{ij} s_j)$: Model alignment, enforcing metacognitive consensus.
\item $-\E_q[\log p(o \mid \cdot)]$: Observation likelihood, grounding beliefs in sensory data.
\end{enumerate}

% ============================================================================
\section{Conditional Uniqueness of the Forward KL Divergence}
\label{sec:uniqueness}
% ============================================================================

The appearance of the forward KL divergence $D_{\KL}(q_i \| \Omega_{ij} q_j)$ rather than the reverse KL or some other divergence is not arbitrary. We now establish that, within a well-defined class of variational problems, the forward KL is the unique divergence consistent with exponential family closure and dual consistency of attention weights.

\subsection{The Variational Problem}

Each agent $i$ minimizes a local free energy functional:
\begin{equation}
F_i[\beta_i] = \min_{q_i} \left\{ D_{\KL}(q_i \| p_i) + \sum_{j \neq i} \beta_{ij} \mathcal{D}(q_i, q_j) \right\},
\label{eq:local_fe}
\end{equation}
where $\mathcal{D}$ is some divergence measure to be determined, and $\beta_{ij} \geq 0$ with $\sum_j \beta_{ij} = 1$ are attention weights.

We seek divergences $\mathcal{D}$ satisfying three requirements:
\begin{enumerate}
\item \textbf{Locality}: $\mathcal{D}$ is a local $f$-divergence of the form $\mathcal{D}(q_i, q_j) = \int q_i(c) f\left(\frac{q_i(c)}{\Omega_{ij}(c) q_j(c)}\right) dc$.
\item \textbf{Linear coupling}: $\mathcal{D}$ enters linearly in \eqref{eq:local_fe}.
\item \textbf{Exponential family closure}: The stationary solution $q_i^*$ remains in the exponential family (log-linear form).
\end{enumerate}

\subsection{The Forward KL Solution}

For $\mathcal{D}(q_i, q_j) = D_{\KL}(q_i \| \Omega_{ij} q_j)$, the functional derivative is:
\begin{equation}
\frac{\delta D_{\KL}(q_i \| \Omega_{ij} q_j)}{\delta q_i(c)} = \log \frac{q_i(c)}{\Omega_{ij}(c) q_j(c)} + 1.
\end{equation}

The stationarity condition for \eqref{eq:local_fe} is:
\begin{equation}
H_i(c) + \log q_i(c) + \sum_j \beta_{ij} \left[\log \frac{q_i(c)}{\Omega_{ij}(c) q_j(c)} + 1\right] = \lambda_i,
\end{equation}
where $H_i(c) = -\log p_i(c)$ is the local ``energy'' and $\lambda_i$ enforces normalization.

Rearranging and solving for $q_i(c)$:
\begin{equation}
\boxed{q_i^*(c) = \frac{1}{Z_i} e^{-H_i(c)/2} \prod_j \left[\Omega_{ij}(c) q_j(c)\right]^{\beta_{ij}/2}}
\label{eq:geometric_mean}
\end{equation}

This is a geometric mean of the prior $p_i \propto e^{-H_i}$ and the transported neighbor beliefs $\Omega_{ij} q_j$, weighted by attention. The solution is log-linear in these constituents, preserving exponential family form.

\subsection{Failure of Alternative Divergences}

\paragraph{Reverse KL.} For $\mathcal{D}(q_i, q_j) = D_{\KL}(\Omega_{ij} q_j \| q_i)$, the functional derivative is:
\begin{equation}
\frac{\delta D_{\KL}(\Omega_{ij} q_j \| q_i)}{\delta q_i(c)} = -\frac{\Omega_{ij}(c) q_j(c)}{q_i(c)}.
\end{equation}

The stationarity condition becomes:
\begin{equation}
H_i(c) + \log q_i(c) - \sum_j \beta_{ij} \frac{\Omega_{ij}(c) q_j(c)}{q_i(c)} = \text{const}.
\end{equation}

This introduces $1/q_i$ terms, yielding a transcendental equation without closed-form solution. The exponential family is not closed under this optimization.

\paragraph{Symmetric KL.} The symmetrized divergence $\frac{1}{2}[D_{\KL}(q_i \| \Omega_{ij} q_j) + D_{\KL}(\Omega_{ij} q_j \| q_i)]$ mixes $\log q_i$ and $1/q_i$ terms, again breaking log-linearity.

\subsection{The Conditional Uniqueness Theorem}

\begin{theorem}[Conditional Uniqueness]
\label{thm:uniqueness}
Let $\mathcal{D}(q_i, q_j)$ be any local $f$-divergence that enters linearly in \eqref{eq:local_fe}, and suppose the stationary distribution $q_i^*$ is log-linear in $\{H_i, \Omega_{ij} q_j\}$. Then the following are equivalent:
\begin{enumerate}
\item $q_i^*$ has the geometric-mean form \eqref{eq:geometric_mean};
\item $\mathcal{D}(q_i, q_j) = D_{\KL}(q_i \| \Omega_{ij} q_j)$;
\item The marginal cost of attention satisfies $C_{ij} := \frac{\partial F_i}{\partial \beta_{ij}} = D_{\KL}(q_i^* \| \Omega_{ij} q_j)$.
\end{enumerate}
\end{theorem}

\begin{proof}
$(2) \Rightarrow (1)$: Direct calculation as shown above.

$(1) \Rightarrow (2)$: Assume $q_i^*$ has form \eqref{eq:geometric_mean}. Substituting into the stationarity condition and matching terms, we find that the functional derivative of $\mathcal{D}$ must equal $\log(q_i / \Omega_{ij} q_j) + 1$. Integrating with the constraint $\mathcal{D}(q, q) = 0$ uniquely determines $\mathcal{D} = D_{\KL}$.

$(2) \Leftrightarrow (3)$: By the envelope theorem, at the stationary point $q_i^*$:
\begin{equation}
\frac{\partial F_i}{\partial \beta_{ij}} = \mathcal{D}(q_i^*, q_j).
\end{equation}
This identifies the marginal cost of increasing attention to agent $j$ with the divergence between the updated belief and the transported neighbor. Only for forward KL does this equal $D_{\KL}(q_i^* \| \Omega_{ij} q_j)$.
\end{proof}

\begin{remark}[Information-Geometric Interpretation]
The forward KL divergence is the Bregman divergence generated by the negative entropy potential $\Phi(q) = \int q \log q$. Its Hessian is the Fisher information matrix, which induces the Fisher-Rao metric on the statistical manifold. The uniqueness theorem reflects deep connections between variational inference and information geometry: the forward KL is distinguished by its relationship to the natural exponential family structure.
\end{remark}

% ============================================================================
\section{Extension to Exponential Families}
\label{sec:exponential_families}
% ============================================================================

The derivations above specialize to Gaussian distributions for concreteness, but the essential structure extends to arbitrary exponential families.

\subsection{Exponential Family Distributions}

An exponential family has the canonical form:
\begin{equation}
p(x; \theta) = h(x) \exp\left(\theta^\T T(x) - A(\theta)\right),
\end{equation}
where:
\begin{itemize}
\item $\theta \in \Theta \subseteq \R^d$ are natural parameters,
\item $T: \mathcal{X} \to \R^d$ are sufficient statistics,
\item $A: \Theta \to \R$ is the log-partition function (cumulant generating function),
\item $h: \mathcal{X} \to \R_+$ is the base measure.
\end{itemize}

\begin{example}[Gaussian as Exponential Family]
For $\N(\mu, \Sigma)$, the natural parameters are $\theta = (\Sigma^{-1}\mu, -\frac{1}{2}\Sigma^{-1})$ and sufficient statistics are $T(x) = (x, xx^\T)$. The log-partition function is $A(\theta) = -\frac{1}{4}\theta_1^\T \theta_2^{-1} \theta_1 - \frac{1}{2}\log|-2\theta_2| + \frac{d}{2}\log(2\pi)$.
\end{example}

\begin{example}[Categorical Distribution]
For a distribution over $K$ categories with probabilities $\pi = (\pi_1, \ldots, \pi_K)$, the natural parameters are $\theta_k = \log(\pi_k / \pi_K)$ for $k < K$, with sufficient statistics $T_k(x) = \mathbbm{1}_{x = k}$ and log-partition $A(\theta) = \log(1 + \sum_{k<K} e^{\theta_k})$.
\end{example}

\subsection{Fisher-Rao Metric for Exponential Families}

The Fisher information matrix for an exponential family is:
\begin{equation}
G(\theta) = \nabla^2 A(\theta) = \text{Cov}_{p_\theta}[T(X)].
\end{equation}

This is the Hessian of the log-partition function, which equals the covariance matrix of sufficient statistics under $p_\theta$. The Fisher-Rao metric is:
\begin{equation}
ds^2 = d\theta^\T G(\theta) d\theta.
\end{equation}

\subsection{KL Divergence as Bregman Divergence}

For exponential families, the KL divergence admits the Bregman divergence representation:
\begin{equation}
D_{\KL}(p_{\theta_1} \| p_{\theta_2}) = A(\theta_2) - A(\theta_1) - \nabla A(\theta_1)^\T (\theta_2 - \theta_1) = D_A(\theta_2 \| \theta_1),
\end{equation}
where $D_A$ is the Bregman divergence generated by $A$.

This representation makes manifest that KL divergence measures the gap between $A(\theta_2)$ and its first-order Taylor approximation around $\theta_1$. The convexity of $A$ (guaranteed for exponential families) ensures $D_{\KL} \geq 0$.

\subsection{Generalization of the Variational Free Energy}

The variational free energy \eqref{eq:vfe_final} generalizes immediately:
\begin{equation}
\F[\{q_i\}, \{s_i\}] = \sum_i D_{\KL}(q_i \| p_i) + \sum_{i,j} \beta_{ij} D_{\KL}(q_i \| \Omega_{ij} q_j) + \ldots
\end{equation}
where now $q_i, p_i$ are members of an arbitrary exponential family, $\Omega_{ij}$ acts on the natural parameters via the group representation, and KL divergence is computed via the Bregman formula.

The conditional uniqueness theorem (Theorem~\ref{thm:uniqueness}) holds verbatim: the geometric-mean solution \eqref{eq:geometric_mean} corresponds to linear combination in natural parameter space:
\begin{equation}
\theta_i^* = \frac{1}{2}\theta_{p_i} + \sum_j \frac{\beta_{ij}}{2} \Omega_{ij} \theta_j,
\end{equation}
which remains within the exponential family.

\subsection{Implications}

This generalization has several consequences:
\begin{enumerate}
\item \textbf{Categorical distributions}: Standard transformer attention over discrete tokens fits within this framework, with softmax probabilities as the natural parameterization.
\item \textbf{Directional data}: Von Mises-Fisher distributions on spheres permit modeling embeddings normalized to unit length.
\item \textbf{Count data}: Poisson distributions permit event-based neural coding models.
\item \textbf{Covariance matrices}: Wishart distributions permit modeling second-order statistics.
\end{enumerate}

% ============================================================================
\section{Proper Time as Information-Theoretic Arc Length}
\label{sec:proper_time}
% ============================================================================

A fundamental issue in any dynamical theory of belief is the parameterization of time. Wall-clock time is unsuitable because cognitive processes operate on different timescales for different agents and contexts. We define proper time as the arc length with respect to the Fisher-Rao metric.

\subsection{Definition}

\begin{definition}[Proper Time]
For an infinitesimal change $d\theta$ in natural parameters, the proper time increment is:
\begin{equation}
\boxed{d\tau = \sqrt{d\theta^\T G(\theta) d\theta}}
\end{equation}
where $G(\theta) = \nabla^2 A(\theta)$ is the Fisher information matrix.
\end{definition}

For Gaussians with fixed covariance, this reduces to:
\begin{equation}
d\tau = \sqrt{d\mu^\T \Sigma^{-1} d\mu} = \|d\mu\|_{\Sigma^{-1}}.
\end{equation}

\subsection{Information-Theoretic Interpretation}

To second order, the KL divergence between nearby distributions satisfies:
\begin{equation}
D_{\KL}(p_\theta \| p_{\theta + d\theta}) = \frac{1}{2} d\theta^\T G(\theta) d\theta + O(\|d\theta\|^3) = \frac{1}{2} d\tau^2 + O(\|d\theta\|^3).
\end{equation}

Thus proper time measures accumulated information change: $d\tau^2 \approx 2 D_{\KL}$. This justifies the interpretation of proper time as ``a difference which makes a difference''---the amount of distinguishable information gained or lost.

\subsection{Scale Dependence}

The proper time increment depends on both the magnitude of parameter change and the local precision:
\begin{itemize}
\item \textbf{High-precision agent} ($\Sigma$ small, $\Sigma^{-1}$ large): A small change in $\mu$ corresponds to large proper time. Each update is cognitively significant.
\item \textbf{Low-precision agent} ($\Sigma$ large, $\Sigma^{-1}$ small): The same parametric change corresponds to small proper time. Updates are relatively insignificant.
\end{itemize}

This asymmetry has physical content: a single bit of information is enormous for a simple, low-dimensional agent but imperceptible for a complex, high-dimensional one.

\subsection{Invariance}

Proper time is invariant under reparameterization of the belief space. If $\eta = \eta(\theta)$ is a smooth reparameterization, then:
\begin{equation}
d\tau^2 = d\theta^\T G_\theta d\theta = d\eta^\T G_\eta d\eta,
\end{equation}
where $G_\eta = J^\T G_\theta J$ and $J = \partial\theta/\partial\eta$ is the Jacobian. This is the defining property of a Riemannian metric.

% ============================================================================
\section{Fisher Metric versus Dynamical Mass}
\label{sec:mass_matrix}
% ============================================================================

A crucial distinction must be maintained between two related but distinct quantities: the Fisher-Rao metric, which defines the intrinsic geometry of the statistical manifold, and the Hessian of the free energy, which defines the effective mass governing belief dynamics.

\subsection{The Fisher-Rao Metric}

The Fisher-Rao metric $G(\theta)$ is the unique (up to scaling) Riemannian metric on a statistical manifold that is invariant under sufficient statistics. For Gaussians, $G_{\mu\mu} = \Sigma^{-1}$.

This metric determines:
\begin{itemize}
\item Proper time (arc length)
\item Natural gradients: $\nabla_{\text{nat}} f = G^{-1} \nabla f$
\item Geodesics (shortest paths between distributions)
\item Intrinsic curvature of the manifold
\end{itemize}

\subsection{The Mass Matrix}

In a Hamiltonian formulation of belief dynamics (adopted as an ansatz motivated by the geometric structure), the effective mass is the Hessian of the free energy:
\begin{equation}
\mathbf{M} = \frac{\partial^2 \F}{\partial \xi \partial \xi^\T},
\end{equation}
where $\xi = (\mu_1, \ldots, \mu_N, \ldots)$ collects all mean parameters.

Unlike the Fisher metric, this Hessian incorporates contributions from all terms in the free energy, not just the intrinsic geometry.

\subsection{Contributions to Dynamical Mass}

From the companion manuscript on belief inertia, the mass matrix for agent $i$ decomposes as:
\begin{equation}
\boxed{
\mathbf{M}_i = \underbrace{\Sigma_{p,i}^{-1}}_{\text{prior precision}} + \underbrace{\Lambda_{o,i}}_{\text{observation precision}} + \underbrace{\sum_k \beta_{ik} \tilde{\Lambda}_{q,k}}_{\text{incoming social}} + \underbrace{\sum_j \beta_{ji} \Lambda_{q,i}}_{\text{outgoing social}}
}
\end{equation}

where:
\begin{itemize}
\item $\Sigma_{p,i}^{-1}$: Prior precision---resistance to deviating from innate expectations.
\item $\Lambda_{o,i} = H^\T R^{-1} H$ for observation model $p(o|k) = \N(o; Hk, R)$: Observation precision---grounding through sensory data.
\item $\sum_k \beta_{ik} \tilde{\Lambda}_{q,k}$: Incoming social precision---being pulled toward confident neighbors.
\item $\sum_j \beta_{ji} \Lambda_{q,i}$: Outgoing social precision---recoil from exerting influence on others.
\end{itemize}

\subsection{The Distinction}

\begin{tcolorbox}[colback=yellow!5,colframe=orange!75,title=Key Distinction]
\begin{itemize}
\item \textbf{Fisher-Rao metric} $G = \Sigma^{-1}$: Intrinsic geometry of the statistical manifold, independent of the free energy functional.
\item \textbf{Mass matrix} $\mathbf{M} = \partial^2 \F / \partial \xi^2$: Total resistance to belief change, incorporating all sources of constraint.
\end{itemize}
The mass matrix includes the Fisher metric (via the KL terms) but also incorporates prior, observation, and social contributions. An agent can have high Fisher information (precise beliefs) but low dynamical mass (weak priors, noisy observations, weak social coupling), or vice versa.
\end{tcolorbox}

\begin{remark}[Caveat]
The identification of the Hessian with dynamical mass assumes a Hamiltonian formulation of belief dynamics. This is an ansatz, not a derivation from first principles. The Fisher metric provides geometry; it does not by itself imply that beliefs evolve according to Hamiltonian mechanics. The framework's predictions in the overdamped limit (gradient flow) depend only on the geometry, while underdamped predictions (oscillation, overshooting) depend on the ansatz.
\end{remark}

% ============================================================================
\section{Correspondence with Continuous Thought Machines}
\label{sec:ctm_correspondence}
% ============================================================================

We now develop the correspondence between the gauge-theoretic framework and CTM architecture. The correspondence is not an identity but rather a structural mapping that illuminates both frameworks.

\subsection{Neuron-Level Models and Local Metrics}

In CTM, each neuron $d$ possesses private weights $W_d$ that process its history of pre-activations. In the gauge-theoretic framework, each agent $i$ possesses a local metric $G_i$ on its belief space.

\begin{proposition}[NLM as Local Metric]
If neuron $d$ maps inputs $x \in \R^n$ to a belief $q_d(z; x) = \N(W_d x, \Sigma_d)$ over latent states $z$, then the Fisher information with respect to the input $x$ is:
\begin{equation}
G_d^{(x)} = W_d^\T \Sigma_d^{-1} W_d.
\end{equation}
\end{proposition}

\begin{proof}
The mean of the induced distribution is $\mu(x) = W_d x$. By the chain rule for Fisher information:
\begin{equation}
G^{(x)} = \left(\frac{\partial \mu}{\partial x}\right)^\T G^{(\mu)} \left(\frac{\partial \mu}{\partial x}\right) = W_d^\T \Sigma_d^{-1} W_d.
\end{equation}
\end{proof}

The NLM weights $W_d$ thus define a neuron-specific metric on input space. Neurons with large $\|W_d\|$ are ``high-precision'' and experience rapid proper time for fixed input changes.

\subsection{Synchronization and KL Alignment}

CTM uses temporal correlation between neuron activations as its synchronization representation:
\begin{equation}
S_{ij} = \text{corr}(x_i, x_j) = \frac{\text{Cov}(x_i, x_j)}{\sigma_i \sigma_j}.
\end{equation}

The gauge-theoretic framework uses KL-based attention:
\begin{equation}
\beta_{ij} = \softmax_j\left(-\frac{D_{\KL}(q_i \| \Omega_{ij} q_j)}{\kappa}\right).
\end{equation}

These are connected through mutual information.

\begin{lemma}[Mutual Information as KL Divergence]
For any joint distribution $p_{XY}$ with marginals $p_X$, $p_Y$:
\begin{equation}
I(X; Y) = D_{\KL}(p_{XY} \| p_X \otimes p_Y).
\end{equation}
\end{lemma}

\begin{lemma}[Correlation and Mutual Information]
For jointly Gaussian $(X, Y)$ with correlation coefficient $\rho$:
\begin{equation}
I(X; Y) = -\frac{1}{2} \log(1 - \rho^2).
\end{equation}
\end{lemma}

\begin{proposition}[Synchronization-KL Correspondence]
For jointly Gaussian activations with small correlation, synchronization and KL-based alignment are related by:
\begin{equation}
|S_{ij}| \approx \sqrt{2 I(x_i; x_j)} = \sqrt{2 D_{\KL}(p_{ij} \| p_i \otimes p_j)}.
\end{equation}
\end{proposition}

\begin{proof}
From $I(X;Y) = -\frac{1}{2}\log(1 - \rho^2)$, we have $\rho^2 = 1 - e^{-2I}$. For small $I$: $\rho^2 \approx 2I$, hence $|\rho| \approx \sqrt{2I}$.
\end{proof}

Both frameworks thus implement attention as a function of information-theoretic distance: low divergence (high mutual information, high correlation) yields high attention weight.

\subsection{Adaptive Computation and Proper Time}

CTM employs adaptive computation: internal iterations continue until a confidence criterion is satisfied (lowest loss combined with highest certainty).

In the gauge-theoretic framework, the natural stopping criterion is stabilization of proper time:
\begin{equation}
\frac{d\tau}{dt} = \sqrt{\dot{\mu}^\T \Sigma^{-1} \dot{\mu}} < \epsilon.
\end{equation}

When beliefs stabilize ($\dot{\mu} \to 0$), proper time accumulation ceases. When beliefs continue evolving, proper time accumulates. This provides a principled adaptive computation rule: iterate until the rate of information change falls below threshold.

The correspondence with CTM's certainty criterion follows from the observation that high certainty corresponds to high precision ($\Sigma^{-1}$ large). When precision is high and stable, even small belief changes produce significant proper time increments, making stabilization of $d\tau/dt$ a sensitive indicator of convergence.

\subsection{Gauge Covariance and Correlation Invariance}

The gauge-theoretic framework features explicit transport operators $\Omega_{ij}$ ensuring that comparisons between agents are made in aligned coordinates.

CTM has no explicit gauge structure. However, correlation is invariant under several transformations:
\begin{itemize}
\item Translations: $x \mapsto x + c$ leaves correlation unchanged.
\item Scalings: $x \mapsto \alpha x$ leaves correlation unchanged.
\item Orthogonal transformations: $x \mapsto Rx$ for $R \in O(n)$ preserves correlation magnitude.
\end{itemize}

These invariances correspond to a trivial gauge structure where $\Omega_{ij} = I$ for all pairs. In the language of the transformer limit derived in the companion manuscript, CTM operates in the flat bundle regime where all agents share a global frame.

\subsection{Summary of Correspondences}

\begin{table}[h]
\centering
\caption{Structural correspondence between gauge theory and CTM}
\label{tab:correspondence}
\begin{tabular}{@{}lll@{}}
\toprule
\textbf{Gauge-Theoretic Framework} & \textbf{CTM} & \textbf{Nature} \\
\midrule
Statistical manifold $\M$ & Neuron state space & Structural \\
Fisher metric $G_i = W_i^\T \Sigma_i^{-1} W_i$ & NLM private weights & Exact \\
$D_{\KL}(q_i \| \Omega_{ij} q_j)$ & Synchronization $S_{ij}$ & Approximate \\
$\beta_{ij} = \softmax(-\KL/\kappa)$ & Attention from sync & Structural \\
Proper time $d\tau = \|d\mu\|_{\Sigma^{-1}}$ & Adaptive iteration & Interpretive \\
Mass matrix $\mathbf{M} = \partial^2 \F$ & Neuron stability & Interpretive \\
Gauge transport $\Omega_{ij}$ & Implicit (correlation invariance) & Trivial limit \\
\bottomrule
\end{tabular}
\end{table}

% ============================================================================
\section{Gaps, Assumptions, and Limitations}
\label{sec:gaps}
% ============================================================================

The correspondence developed above is not an equivalence. We identify the assumptions required and limitations encountered.

\subsection{Gaussian Assumption}

The explicit calculations assume Gaussian distributions. The extension to general exponential families (Section~\ref{sec:exponential_families}) shows this is not essential to the structure, but CTM activations may not be well-modeled by any exponential family due to nonlinearities.

\textbf{Status}: The structural correspondence (KL $\leftrightarrow$ mutual information $\leftrightarrow$ correlation) holds for Gaussians. For non-Gaussian activations, the quantitative relationship changes but the qualitative correspondence (information-theoretic distance $\leftrightarrow$ attention) persists.

\subsection{Temporal vs. Instantaneous}

The gauge-theoretic framework is formulated for instantaneous beliefs $q(t)$; CTM operates on temporal histories of activations.

\textbf{Bridge}: Define temporal distributions over length-$T$ trajectories. The relevant metric becomes an average over the trajectory:
\begin{equation}
G^{\text{temporal}} = \frac{1}{T} \sum_{t=1}^T G(\theta(t)).
\end{equation}
For stationary processes, this reduces to the instantaneous metric multiplied by effective degrees of freedom.

\subsection{Explicit vs. Implicit Optimization}

The gauge-theoretic framework explicitly minimizes the free energy $\F$ via gradient descent:
\begin{equation}
\frac{d\theta_i}{d\tau} = -G^{-1} \nabla_\theta \F.
\end{equation}

CTM uses synchronization as a representation for prediction, with gradients flowing through the synchronization-to-output pathway. This creates implicit rather than explicit alignment pressure.

\textbf{Resolution}: Gradient descent on a task loss $\mathcal{L}(\hat{y}(S))$ where predictions depend on synchronization $S$ induces gradients:
\begin{equation}
\frac{\partial \mathcal{L}}{\partial W_d} = \frac{\partial \mathcal{L}}{\partial \hat{y}} \frac{\partial \hat{y}}{\partial S} \frac{\partial S}{\partial a} \frac{\partial a}{\partial W_d}.
\end{equation}
These gradients encourage synchronization patterns that support accurate prediction---functionally similar to explicit free energy minimization, but learned rather than prescribed.

\subsection{The Hamiltonian Ansatz}

The identification of the free energy Hessian with dynamical mass assumes Hamiltonian belief dynamics. This is geometrically motivated but not derived from first principles. Predictions depending on underdamped dynamics (oscillation, overshooting) are contingent on this ansatz; predictions depending only on gradient flow are geometrically necessary.

\subsection{Gauge Structure}

CTM operates in a trivial gauge regime ($\Omega_{ij} = I$). The full richness of gauge-theoretic structure---non-trivial transport, curvature, topological effects---is absent. This limits the correspondence to the flat-bundle limit of the general framework.

% ============================================================================
\section{Conclusion}
\label{sec:conclusion}
% ============================================================================

We have developed a mathematical correspondence between gauge-theoretic variational inference and the Continuous Thought Machine architecture. The principal findings are:

\begin{enumerate}
\item The gauge-theoretic variational free energy derives from a normalized generative model with auxiliary agreement variables. The forward KL divergence emerges as the unique coupling consistent with exponential family closure.

\item CTM's neuron-level models correspond to local Fisher metrics induced by neuron-specific weight matrices.

\item CTM's synchronization representation corresponds to KL-based belief alignment through the mutual information identity $I(X;Y) = D_{\KL}(p_{XY} \| p_X p_Y)$.

\item CTM's adaptive computation admits interpretation as proper time stabilization, where iteration continues until the rate of information-theoretic distance traversed falls below threshold.

\item The Fisher-Rao metric (intrinsic geometry) must be distinguished from the free energy Hessian (dynamical mass): the former defines proper time, the latter incorporates all constraints on belief change.

\item The framework extends to arbitrary exponential families, suggesting the mathematical structure is not specific to Gaussian assumptions.
\end{enumerate}

The correspondence is structural rather than exact. CTM was not designed with these principles in mind, and operates in a trivial-gauge, implicit-optimization regime that represents a limiting case of the general framework. Nevertheless, the correspondence suggests that information geometry provides a principled foundation for understanding why architectures like CTM exhibit their characteristic behaviors, and points toward explicit implementations that might inherit the full richness of the gauge-theoretic structure.

% ============================================================================
% REFERENCES
% ============================================================================
\begin{thebibliography}{99}

\bibitem{darlow2025ctm}
L.~Darlow, C.~Regan, S.~Risi, J.~Seely, and L.~Jones.
\newblock Continuous Thought Machines.
\newblock \emph{arXiv preprint arXiv:2505.05522}, 2025.

\bibitem{amari1998natural}
S.~Amari.
\newblock Natural gradient works efficiently in learning.
\newblock \emph{Neural Computation}, 10(2):251--276, 1998.

\bibitem{amari2016information}
S.~Amari.
\newblock \emph{Information Geometry and Its Applications}.
\newblock Springer, 2016.

\bibitem{friston2010free}
K.~Friston.
\newblock The free-energy principle: a unified brain theory?
\newblock \emph{Nature Reviews Neuroscience}, 11(2):127--138, 2010.

\bibitem{ay2017information}
N.~Ay, J.~Jost, H.~V.~L\^e, and L.~Schwachh\"ofer.
\newblock \emph{Information Geometry}.
\newblock Springer, 2017.

\bibitem{nakahara2003geometry}
M.~Nakahara.
\newblock \emph{Geometry, Topology and Physics}.
\newblock CRC Press, 2003.

\bibitem{cover2006elements}
T.~M.~Cover and J.~A.~Thomas.
\newblock \emph{Elements of Information Theory}.
\newblock Wiley, 2nd edition, 2006.

\end{thebibliography}

\end{document}
