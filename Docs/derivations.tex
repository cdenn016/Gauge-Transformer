% Detailed Derivations: Classical Models as Limiting Cases of Hamiltonian VFE
% For Sociology Manuscript Submission
% Author: Chris Denning
% Date: December 2025

\documentclass[11pt]{article}
\usepackage{amsmath,amssymb,amsthm}
\usepackage{geometry}
\usepackage{hyperref}
\usepackage{enumitem}
\usepackage{booktabs}

\geometry{margin=1in}

\newtheorem{proposition}{Proposition}
\newtheorem{theorem}{Theorem}
\newtheorem{lemma}{Lemma}
\newtheorem{corollary}{Corollary}
\newcommand{\R}{\mathbb{R}}
\newcommand{\N}{\mathcal{N}}
\newcommand{\KL}{\text{KL}}
\newcommand{\tr}{\text{tr}}

\title{Classical Social Influence Models as Limiting Cases of \\
Hamiltonian Variational Free Energy: \\
Detailed Derivations}

\author{Chris Denning}
\date{December 2025}

\begin{document}

\maketitle

\begin{abstract}
We provide detailed mathematical derivations showing that six major models from sociology, psychology, and network science emerge as special or limiting cases of the Hamiltonian Variational Free Energy framework. These include DeGroot's social learning model, Friedkin-Johnsen opinion dynamics, bounded confidence models, confirmation bias, Social Impact Theory, and echo chamber formation. For each model, we present rigorous proofs or solid approximations with explicit limiting conditions, demonstrating that these seemingly disparate theories represent different dynamical regimes of a unified information-geometric framework.
\end{abstract}

\tableofcontents

\section{Introduction}

Social influence research has produced numerous models describing how individuals update beliefs through social interaction. While empirically successful in their respective domains, these models appear theoretically disconnected. We demonstrate that this fragmentation is illusory: major classical models emerge as limiting cases of a unified framework based on variational free energy minimization on statistical manifolds.

This unification proceeds by analogy to statistical mechanics. Just as thermodynamic phases (solid, liquid, gas) emerge from molecular dynamics under different conditions (temperature, pressure), we show that classical social influence models emerge from information-geometric dynamics under different conditions (friction, uncertainty, attention temperature). The underlying dynamics are always governed by natural gradient descent on the Fisher-Rao manifold, but observable behavior depends on regime parameters.

\subsection{Mathematical Foundations}

\subsubsection{The Variational Free Energy Functional}

Consider $N$ agents, each maintaining a belief distribution $q_i(x) = \N(\mu_i, \Sigma_i)$ and prior $p_i(x) = \N(\mu_{p,i}, \Sigma_{p,i})$ over a latent state $x \in \R^K$. The total variational free energy is:

\begin{align}
F[q, p] &= \sum_i \alpha \int \chi_i(c) \KL(q_i \| p_i) \, dc \label{eq:vfe_self} \\
&\quad + \sum_{i,j} \lambda_\beta \int \chi_{ij}(c) \beta_{ij}(c) \KL(q_i \| \Omega_{ij}[q_j]) \, dc \label{eq:vfe_belief} \\
&\quad + \sum_{i,j} \lambda_\gamma \int \chi_{ij}(c) \gamma_{ij}(c) \KL(p_i \| \Omega_{ij}[p_j]) \, dc \label{eq:vfe_prior} \\
&\quad - \sum_i \lambda_{\text{obs}} \int \chi_i(c) \mathbb{E}_q[\log p(o|x)] \, dc \label{eq:vfe_obs}
\end{align}

Each term serves a distinct functional role in belief dynamics. The self-coupling term \eqref{eq:vfe_self} penalizes deviation from prior expectations, with the strength parameter $\alpha$ controlling resistance to belief change; this captures the cognitive cost of abandoning established world models. The belief alignment term \eqref{eq:vfe_belief} encourages agents to align their current beliefs with those of their neighbors, where the attention weights $\beta_{ij}(c)$ are computed dynamically via softmax (detailed below) rather than fixed exogenously. The prior alignment term \eqref{eq:vfe_prior} operates analogously on priors themselves, capturing slower cultural or normative alignment with the softmax-based weights $\gamma_{ij}(c)$. Finally, the observation term \eqref{eq:vfe_obs} grounds beliefs in sensory evidence by penalizing distributions that fail to explain observed data $o$.

The spatial integration weights $\chi_i(c) \in [0,1]$ represent agent $i$'s epistemic support---the regions of state space where they maintain active beliefs---and $\chi_{ij} = \chi_i \cdot \chi_j$ captures the overlap between agents' cognitive domains.

\subsubsection{Softmax Attention Mechanism}

The belief alignment weights are computed dynamically as:
\begin{equation}
\beta_{ij}(c) = \frac{\exp\left(-\KL(q_i(c) \| \Omega_{ij}[q_j(c)]) / \kappa_\beta\right)}{\sum_k \exp\left(-\KL(q_i(c) \| \Omega_{ik}[q_k(c)]) / \kappa_\beta\right)}
\label{eq:softmax_attention}
\end{equation}

This softmax creates \emph{homophilic attention}: agents with similar beliefs (low KL divergence) receive high attention, while dissimilar agents are ignored. The temperature parameter $\kappa_\beta > 0$ controls sharpness: $\kappa_\beta \to 0$ gives winner-take-all attention, while $\kappa_\beta \to \infty$ gives uniform attention.

The transport operator $\Omega_{ij} \in SO(K)$ accounts for gauge transformations (different reference frames between agents). For flat manifolds with shared coordinates, $\Omega_{ij} = I$.

\subsubsection{Fisher-Rao Mass Matrix (Epistemic Inertia)}

The natural gradient descent dynamics on the statistical manifold are governed by the inverse Fisher metric (mass matrix):
\begin{equation}
M_i(\theta) = \Sigma_{p,i}^{-1} + \Sigma_{o,i}^{-1} + \sum_j \beta_{ij}(\theta) \, \Omega_{ij} \Sigma_{q,j}^{-1} \Omega_{ij}^T + \sum_j \beta_{ji}(\theta) \, \Sigma_{q,i}^{-1}
\label{eq:mass_matrix}
\end{equation}

This matrix comprises four physically interpretable contributions. The first term, $\Sigma_{p,i}^{-1}$, represents \emph{prior mass}---the resistance to belief change arising from established expectations, where high prior precision (low $\Sigma_p$) creates correspondingly high inertia. The second term, $\Sigma_{o,i}^{-1}$, captures \emph{observation mass}, reflecting how precise sensory evidence anchors beliefs against social pressure; agents with strong perceptual grounding are less susceptible to influence. The third term, $\sum_j \beta_{ij} \Omega_{ij} \Sigma_{q,j}^{-1} \Omega_{ij}^T$, constitutes the \emph{incoming social mass}---the accumulated inertia from attending to confident neighbors, where being pulled toward precise sources paradoxically increases one's own resistance to further change. Most notably, the fourth term, $\sum_j \beta_{ji} \Sigma_{q,i}^{-1}$, represents \emph{outgoing social mass}: the recoil from exerting influence on others. Agents with many followers (high $\sum_j \beta_{ji}$) acquire epistemic rigidity not through irrationality but through geometric necessity---the ``burden of leadership'' that renders opinion leaders resistant to revision.

\subsubsection{Dynamics: Overdamped vs. Hamiltonian Regimes}

The framework admits two dynamical regimes:

\paragraph{Overdamped (gradient flow):} When friction $\gamma \to \infty$, momentum dissipates instantly:
\begin{equation}
\frac{d\mu_i}{dt} = -M_i^{-1} \nabla_{\mu_i} F
\label{eq:overdamped}
\end{equation}

This is \emph{natural gradient descent} on the statistical manifold. The mass matrix $M^{-1}$ appears even in the overdamped limit, distinguishing natural from standard Euclidean gradient descent.

\paragraph{Hamiltonian (underdamped):} When friction $\gamma \to 0$, the system exhibits inertial dynamics:
\begin{align}
\frac{d\mu_i}{dt} &= M_i^{-1} \pi_{\mu,i} \label{eq:hamiltonian_position} \\
\frac{d\pi_{\mu,i}}{dt} &= -\nabla_{\mu_i} F - \Gamma_{ijk} \pi^j \pi^k - \gamma \pi_{\mu,i} \label{eq:hamiltonian_momentum}
\end{align}

Here $\pi_{\mu,i}$ is conjugate momentum, and $\Gamma_{ijk}$ are Christoffel symbols (geodesic corrections from metric curvature). This regime allows belief oscillations, overshooting, and non-monotonic convergence.

Classical models correspond to the \emph{overdamped regime} with various limiting conditions on parameters $(\alpha, \lambda_\beta, \kappa_\beta, \Sigma_i)$. The Hamiltonian regime generates novel predictions beyond classical theories.

\section{Derivation 1: DeGroot Social Learning (Rigorous)}

\subsection{Classical Formulation}

DeGroot's model (1974) describes social learning as iterative averaging of neighbors' beliefs:
\begin{equation}
x_i(t+1) = \sum_j w_{ij} x_j(t)
\label{eq:degroot_classical}
\end{equation}

where $W = [w_{ij}]$ is a row-stochastic matrix ($\sum_j w_{ij} = 1$) representing social influence weights. Under mild conditions, beliefs converge to a consensus determined by the network structure.

\subsection{Sociological Context}

DeGroot's model captures the fundamental sociological insight that individuals update beliefs by averaging opinions from their social network. It has been applied to jury deliberation, scientific consensus formation, and organizational decision-making. However, the model treats influence weights $w_{ij}$ as exogenous and fixed, providing no mechanism for how attention emerges from belief similarity.

\subsection{Derivation from VFE Framework}

\begin{proposition}[DeGroot as VFE Limit]
The DeGroot update rule \eqref{eq:degroot_classical} emerges from the VFE framework under the following limiting conditions:
\begin{enumerate}[label=(\roman*)]
\item Overdamped dynamics: $\gamma \to \infty$
\item Low uncertainty: $\Sigma_i \to \sigma^2 I$ with $\sigma^2$ small
\item Flat manifold: $\Omega_{ij} = I$ (shared reference frames)
\item No self-coupling: $\alpha = 0$
\item No prior alignment: $\lambda_\gamma = 0$
\item No observations: $\lambda_{\text{obs}} = 0$
\item Fixed attention: $\beta_{ij} = w_{ij}$ (constant, not softmax)
\end{enumerate}
\end{proposition}

\begin{proof}
\textbf{Step 1: Simplify VFE.}
Under conditions (iii)-(vii), the free energy \eqref{eq:vfe_belief} reduces to:
\begin{equation}
F[\mu] = \lambda_\beta \sum_{i,j} w_{ij} \int \KL(\N(\mu_i, \sigma^2 I) \| \N(\mu_j, \sigma^2 I)) \, dc
\end{equation}

For Gaussian distributions with equal covariances, the KL divergence is:
\begin{equation}
\KL(\N(\mu_i, \sigma^2 I) \| \N(\mu_j, \sigma^2 I)) = \frac{\|\mu_i - \mu_j\|^2}{2\sigma^2}
\end{equation}

Thus:
\begin{equation}
F[\mu] = \frac{\lambda_\beta}{2\sigma^2} \sum_{i,j} w_{ij} \|\mu_i - \mu_j\|^2
\label{eq:degroot_vfe_simplified}
\end{equation}

\textbf{Step 2: Compute gradient.}
Taking the gradient with respect to $\mu_i$:
\begin{align}
\nabla_{\mu_i} F &= \frac{\lambda_\beta}{\sigma^2} \sum_j w_{ij} (\mu_i - \mu_j) \\
&= \frac{\lambda_\beta}{\sigma^2} \left[\mu_i \sum_j w_{ij} - \sum_j w_{ij} \mu_j\right] \\
&= \frac{\lambda_\beta}{\sigma^2} \left[\mu_i - \sum_j w_{ij} \mu_j\right] \quad \text{(using row-stochasticity)}
\label{eq:degroot_gradient}
\end{align}

\textbf{Step 3: Apply natural gradient flow.}
In the low-uncertainty limit (condition ii), the mass matrix becomes approximately:
\begin{equation}
M_i \approx \frac{1}{\sigma^2} I
\end{equation}

Natural gradient descent \eqref{eq:overdamped} gives:
\begin{align}
\frac{d\mu_i}{dt} &= -M_i^{-1} \nabla_{\mu_i} F \\
&= -\sigma^2 I \cdot \frac{\lambda_\beta}{\sigma^2} \left(\mu_i - \sum_j w_{ij} \mu_j\right) \\
&= -\lambda_\beta \left(\mu_i - \sum_j w_{ij} \mu_j\right)
\label{eq:degroot_continuous}
\end{align}

\textbf{Step 4: Discretize.}
Applying forward Euler integration with time step $\Delta t = 1/\lambda_\beta$:
\begin{align}
\mu_i(t + \Delta t) &= \mu_i(t) + \Delta t \cdot \frac{d\mu_i}{dt} \\
&= \mu_i(t) - \lambda_\beta \Delta t \left(\mu_i - \sum_j w_{ij} \mu_j\right) \\
&= \mu_i(t) - \left(\mu_i - \sum_j w_{ij} \mu_j\right) \\
&= \sum_j w_{ij} \mu_j(t)
\end{align}

This is exactly the DeGroot update \eqref{eq:degroot_classical}. \qed
\end{proof}

\subsection{What the Unified Framework Adds}

While DeGroot's model emerges cleanly in the overdamped limit with fixed weights, the full VFE framework provides several extensions with sociological significance:

\paragraph{Dynamic attention.} Removing condition (vii) and using softmax attention \eqref{eq:softmax_attention}, influence weights become endogenous:
\begin{equation}
\beta_{ij}(t) = \frac{\exp(-\|\mu_i(t) - \mu_j(t)\|^2 / (2\sigma^2\kappa_\beta))}{\sum_k \exp(-\|\mu_i(t) - \mu_k(t)\|^2 / (2\sigma^2\kappa_\beta))}
\end{equation}

Agents pay more attention to similar others, creating homophily as an emergent property rather than assumption.

\paragraph{Uncertainty dynamics.} Removing condition (ii), beliefs are full distributions $q_i = \N(\mu_i, \Sigma_i)$. Uncertainty can increase or decrease over time, capturing phenomena like pluralistic ignorance or confidence polarization that mean-only models miss.

\paragraph{Epistemic inertia.} When agents receive asymmetric attention (some have many followers), the outgoing social mass term in \eqref{eq:mass_matrix} becomes significant:
\begin{equation}
M_i = \Sigma_p^{-1} + \Sigma_o^{-1} + \sum_j \beta_{ij} \tilde{\Sigma}_{q,j}^{-1} + \sum_j \beta_{ji} \Sigma_{q,i}^{-1}
\end{equation}

where $\tilde{\Sigma}_{q,j}^{-1} = \Omega_{ij}\Sigma_{q,j}^{-1}\Omega_{ij}^T$ is the transported precision. High-attention agents (influencers, opinion leaders) develop higher mass via the $\sum_j \beta_{ji}$ term, making their beliefs more resistant to change---a mechanistic explanation for rigidity in positions of authority.

\paragraph{Underdamped regime.} Removing condition (i) and allowing low friction $\gamma < \gamma_c$, beliefs can overshoot equilibrium and oscillate. This provides a potential mechanism for opinion cycling and instability in rapid social media discourse.

\subsection{Novel Predictions}

The unified framework generates several testable predictions beyond DeGroot's original model. First, influence weights $\beta_{ij}(t)$ should evolve dynamically as beliefs converge, a prediction testable via longitudinal network analysis tracking communication patterns alongside opinion change. Second, agents occupying central network positions---those with many followers---should update their beliefs more slowly than peripheral agents, even when controlling for initial belief extremity; this heterogeneous convergence follows from the asymmetric mass accumulation in \eqref{eq:mass_matrix}. Third, as beliefs align through social learning, agents should report increased subjective confidence (manifested as decreasing $\Sigma_i$), a prediction requiring measurement of confidence intervals rather than point estimates alone.

\section{Derivation 2: Friedkin-Johnsen Opinion Dynamics (Rigorous)}

\subsection{Classical Formulation}

Friedkin and Johnsen (1990) extended DeGroot by introducing ``stubbornness''---attachment to initial opinions:
\begin{equation}
x_i(t+1) = \alpha_i x_i(0) + (1 - \alpha_i) \sum_j w_{ij} x_j(t)
\label{eq:fj_classical}
\end{equation}

where $\alpha_i \in [0,1]$ represents agent $i$'s resistance to social influence. This model better captures polarization and persistent disagreement, as stubborn agents prevent full consensus.

\subsection{Sociological Context}

The Friedkin-Johnsen model addresses a key empirical puzzle: social groups often fail to reach consensus despite dense communication. The stubbornness parameter $\alpha_i$ is typically interpreted as a personality trait or ideological commitment. However, this raises the question: what determines stubbornness, and can it change over time?

\subsection{Derivation from VFE Framework}

\begin{proposition}[Friedkin-Johnsen as VFE Equilibrium]
The Friedkin-Johnsen equilibrium opinions emerge from the VFE framework under DeGroot conditions (i)-(vii) plus:
\begin{enumerate}[label=(\roman*), start=8]
\item Non-zero self-coupling: $\alpha > 0$
\item Fixed priors: $p_i = \N(\mu_i(0), \Sigma_p)$ (initial beliefs)
\end{enumerate}

Moreover, the stubbornness parameter $\alpha_i$ is not exogenous but emerges from prior precision and social context.
\end{proposition}

\begin{proof}
\textbf{Step 1: VFE with self-coupling.}
Including the self-coupling term \eqref{eq:vfe_self}:
\begin{align}
F[\mu] &= \alpha \sum_i \KL(\N(\mu_i, \sigma^2 I) \| \N(\mu_i(0), \Sigma_p)) \\
&\quad + \frac{\lambda_\beta}{2\sigma^2} \sum_{i,j} w_{ij} \|\mu_i - \mu_j\|^2
\end{align}

For small $\sigma^2$, the self-coupling KL divergence approximates:
\begin{equation}
\KL(\N(\mu_i, \sigma^2 I) \| \N(\mu_i(0), \Sigma_p)) \approx \frac{\|\mu_i - \mu_i(0)\|^2}{2\Sigma_p}
\end{equation}

Thus:
\begin{equation}
F[\mu] = \frac{\alpha}{2\Sigma_p} \sum_i \|\mu_i - \mu_i(0)\|^2 + \frac{\lambda_\beta}{2\sigma^2} \sum_{i,j} w_{ij} \|\mu_i - \mu_j\|^2
\label{eq:fj_vfe}
\end{equation}

\textbf{Step 2: Gradient.}
\begin{equation}
\nabla_{\mu_i} F = \frac{\alpha}{\Sigma_p}(\mu_i - \mu_i(0)) + \frac{\lambda_\beta}{\sigma^2}\left(\mu_i - \sum_j w_{ij} \mu_j\right)
\end{equation}

\textbf{Step 3: Natural gradient flow with mass $M_i = \Sigma_p^{-1}$.}
\begin{align}
\frac{d\mu_i}{dt} &= -\Sigma_p \nabla_{\mu_i} F \\
&= -\alpha(\mu_i - \mu_i(0)) - \frac{\lambda_\beta \Sigma_p}{\sigma^2}\left(\mu_i - \sum_j w_{ij} \mu_j\right)
\end{align}

\textbf{Step 4: Equilibrium solution.}
At steady state, $d\mu_i/dt = 0$:
\begin{align}
\alpha(\mu_i - \mu_i(0)) &= -\frac{\lambda_\beta \Sigma_p}{\sigma^2}\left(\mu_i - \sum_j w_{ij} \mu_j\right) \\
\mu_i \left[\alpha + \frac{\lambda_\beta \Sigma_p}{\sigma^2}\right] &= \alpha \mu_i(0) + \frac{\lambda_\beta \Sigma_p}{\sigma^2} \sum_j w_{ij} \mu_j
\end{align}

Dividing both sides by the bracketed term:
\begin{equation}
\mu_i = \frac{\alpha}{\alpha + \lambda_\beta \Sigma_p / \sigma^2} \mu_i(0) + \frac{\lambda_\beta \Sigma_p / \sigma^2}{\alpha + \lambda_\beta \Sigma_p / \sigma^2} \sum_j w_{ij} \mu_j
\end{equation}

Define the \emph{emergent stubbornness}:
\begin{equation}
\alpha_i' = \frac{\alpha}{\alpha + \lambda_\beta \Sigma_p / \sigma^2 \cdot \sum_j w_{ij}}
\label{eq:emergent_stubbornness}
\end{equation}

Then:
\begin{equation}
\mu_i = \alpha_i' \mu_i(0) + (1 - \alpha_i') \sum_j w_{ij} \mu_j
\end{equation}

which matches the Friedkin-Johnsen form \eqref{eq:fj_classical}. \qed
\end{proof}

\subsection{Mechanistic Stubbornness}

The key sociological insight is that stubbornness $\alpha_i'$ in \eqref{eq:emergent_stubbornness} is \emph{not} a fixed personality trait but emerges from two sources:

\paragraph{Prior precision $\Sigma_p^{-1}$.} Agents with strong initial convictions (low $\Sigma_p$) exhibit high stubbornness. This captures ideological commitment or expertise: a climate scientist has high prior precision about global warming, making them resistant to contrarian social influence.

\paragraph{Social coupling strength $\lambda_\beta \sum_j w_{ij}$.} Agents experiencing intense social pressure (many influential neighbors) become \emph{less} stubborn in equilibrium. However, as we show next, this same pressure increases their \emph{inertial mass}, slowing their rate of approach to equilibrium---a subtle but important distinction.

\subsection{Dynamic Stubbornness and Social Context}

Unlike the classical model where $\alpha_i$ is constant, the VFE framework predicts that stubbornness changes with social context. Consider two agents with identical priors but different network positions. Agent A, embedded in a dense network with $\sum_j w_{Aj}$ large, exhibits lower equilibrium stubbornness $\alpha_A'$ due to intense social pressure, yet paradoxically develops higher inertial mass $M_A$ from the attention received. Agent B, occupying a peripheral position with $\sum_j w_{Bj}$ small, maintains higher equilibrium stubbornness $\alpha_B'$ but lower inertial mass $M_B$. This dissociation between equilibrium position and dynamic responsiveness generates the testable prediction that centrally-located agents should express opinions closer to the group mean (low $\alpha'$) while simultaneously updating more slowly over time (high $M$).

\subsection{Novel Predictions}

The framework yields three principal predictions for Friedkin-Johnsen dynamics. Network position effects should manifest as a correlation between stubbornness and network peripherality, controlling for prior beliefs. The distinction between equilibrium displacement (stubbornness) and update rate (inertia) should be empirically separable in time-series data, as these quantities derive from different terms in the framework. Perhaps most importantly, the same individual should exhibit different degrees of stubbornness across different social contexts---for instance, between professional and personal networks---contradicting trait-based theories that treat resistance to influence as a stable personality characteristic.

\section{Derivation 3: Echo Chambers and Polarization (Rigorous)}

\subsection{Phenomenon}

Echo chambers constitute a self-reinforcing dynamical process with four characteristic stages. Initially, individuals preferentially attend to similar others---a phenomenon termed homophily. This selective attention causes in-group beliefs to converge through assimilation, while simultaneously causing out-group beliefs to diverge through polarization. The process culminates in isolation, as cross-group communication declines to negligible levels. This phenomenon has become central to understanding political polarization, radicalization, and the epistemic consequences of algorithmically-mediated social media.

\subsection{Classical Approaches}

Traditional models either \emph{assume} homophily (similarity-based edge formation) or impose exogenous group structure. The VFE framework shows that both homophily and polarization \emph{emerge endogenously} from the attention mechanism without additional assumptions.

\subsection{Derivation from VFE Framework}

\begin{proposition}[Emergent Homophily and Polarization]
Softmax attention \eqref{eq:softmax_attention} automatically creates homophilic coupling. When initial belief distributions are multimodal, this leads to stable polarized states with within-group consensus and cross-group divergence.
\end{proposition}

\begin{proof}[Proof Sketch]
\textbf{Step 1: Softmax creates homophily.}
For Gaussian beliefs with common covariance $\Sigma = \sigma^2 I$:
\begin{equation}
\beta_{ij} = \frac{\exp(-\|\mu_i - \mu_j\|^2 / (2\sigma^2 \kappa_\beta))}{\sum_k \exp(-\|\mu_i - \mu_k\|^2 / (2\sigma^2 \kappa_\beta))}
\end{equation}

Similar beliefs (small $\|\mu_i - \mu_j\|$) yield high $\beta_{ij}$ (strong attention). Dissimilar beliefs (large $\|\mu_i - \mu_j\|$) yield low $\beta_{ij}$ (ignore out-group).

\textbf{Step 2: Positive feedback loop.}
The natural gradient flow is:
\begin{equation}
\frac{d\mu_i}{dt} \propto \sum_j \beta_{ij}(\mu) (\mu_j - \mu_i)
\end{equation}

This creates a positive feedback loop through the following mechanism. Agents with similar beliefs attend preferentially to each other, generating high $\beta_{ij}$. This mutual attention causes their beliefs to converge further, $\mu_i \to \mu_j$, which in turn increases attention still more as KL divergence decreases. The process continues until cross-group attention effectively vanishes, $\beta_{ij}^{\text{cross}} \to 0$, leaving the population fragmented into internally cohesive but mutually isolated clusters.

\textbf{Step 3: Stability of polarized states.}
Consider two groups $A$ and $B$ with means $\mu_A$ and $\mu_B$. Within-group attention is:
\begin{equation}
\beta_{ij} \approx \frac{1}{|A|} \quad \text{for } i, j \in A
\end{equation}

Cross-group attention is approximately:
\begin{equation}
\beta_{ij} \approx 0 \quad \text{for } i \in A, j \in B \quad \text{when } \|\mu_A - \mu_B\| \gg \sigma\sqrt{\kappa_\beta}
\end{equation}

The system decouples into two independent subsystems, each converging internally:
\begin{equation}
\frac{d\mu_i}{dt} \approx \sum_{j \in \text{group}(i)} \beta_{ij} (\mu_j - \mu_i)
\end{equation}

\textbf{Step 4: Critical distance for polarization.}
Polarized state $\{\mu_A, \mu_B\}$ is stable when cross-group attention is negligible:
\begin{equation}
\exp\left(-\frac{\|\mu_A - \mu_B\|^2}{2\sigma^2 \kappa_\beta}\right) \ll 1
\end{equation}

This occurs when:
\begin{equation}
\|\mu_A - \mu_B\|^2 > 2\sigma^2 \kappa_\beta \log N
\label{eq:polarization_threshold}
\end{equation}

where $N$ is the number of agents. \qed
\end{proof}

\subsection{Phase Transition in Polarization}

The stability condition \eqref{eq:polarization_threshold} reveals a \emph{phase transition} in the temperature parameter $\kappa_\beta$:

\paragraph{High temperature ($\kappa_\beta$ large):} Attention is diffuse, cross-group communication persists, system converges to global consensus.

\paragraph{Low temperature ($\kappa_\beta$ small):} Attention is sharp, cross-group communication collapses, system locks into polarized state.

The critical temperature scales with the initial belief separation:
\begin{equation}
\kappa_\beta^{\text{crit}} \sim \frac{\|\mu_A(0) - \mu_B(0)\|^2}{2\sigma^2 \log N}
\end{equation}

This provides a quantitative prediction: polarization is more likely when (i) initial disagreement is large, (ii) uncertainty is low, or (iii) attention is selective (low $\kappa_\beta$).

\subsection{Connection to Filter Bubbles}

Social media platforms that use engagement-based ranking effectively lower $\kappa_\beta$ (sharpen attention toward similar content). The VFE framework predicts this design choice should increase polarization, consistent with empirical observations.

Interventions to reduce polarization should target $\kappa_\beta$: increasing exposure diversity (raising temperature) or increasing epistemic humility (raising uncertainty $\sigma^2$) can prevent polarization phase transition.

\subsection{Novel Predictions}

The framework generates three principal predictions concerning echo chamber dynamics. First, varying attention selectivity experimentally should reveal a sharp phase transition: polarization should emerge abruptly at a critical value of $\kappa_\beta$ rather than gradually, analogous to thermodynamic phase transitions. Second, polarized states should be reversible under appropriate interventions---raising $\kappa_\beta$ through increased cross-group exposure, or increasing epistemic uncertainty $\sigma^2$ through epistemic humility interventions, can in principle dissolve echo chambers by restoring cross-group attention. Third, smaller groups should polarize more easily than larger ones, as the threshold in \eqref{eq:polarization_threshold} scales with $\log N$; this asymmetric polarization may explain why small online communities often exhibit more extreme views than larger populations.

\section{Derivation 4: Bounded Confidence Models (Solid Approximation)}

\subsection{Classical Formulation}

Hegselmann-Krause (2002) and Deffuant et al. (2000) introduced bounded confidence: agents only interact with others within a threshold distance $\epsilon$:
\begin{equation}
x_i(t+1) = \begin{cases}
\text{avg}\{x_j(t) : |x_j(t) - x_i(t)| < \epsilon\} & \text{if } |\{j : |x_j - x_i| < \epsilon\}| > 0 \\
x_i(t) & \text{otherwise}
\end{cases}
\label{eq:hk_classical}
\end{equation}

This generates clustering: opinions fragment into groups separated by more than $\epsilon$, with consensus within each group.

\subsection{Sociological Context}

Bounded confidence models capture the idea that individuals ignore opinions too far from their own---a form of selective exposure or cognitive dissonance avoidance. The threshold $\epsilon$ is typically treated as an exogenous parameter, but sociologically, it should depend on context.

\subsection{Derivation from VFE Framework}

\begin{proposition}[Bounded Confidence as Low-Temperature Limit]
The bounded confidence dynamics approximate the VFE framework in the low-temperature regime $\kappa_\beta \to 0$, with effective threshold:
\begin{equation}
\epsilon_{\text{eff}} \approx \sigma\sqrt{2\kappa_\beta \log N}
\end{equation}
\end{proposition}

\begin{proof}[Proof Sketch]
\textbf{Step 1: Low-temperature softmax.}
As $\kappa_\beta \to 0$, the softmax \eqref{eq:softmax_attention} becomes increasingly sharp:
\begin{equation}
\beta_{ij} = \frac{\exp(-\|\mu_i - \mu_j\|^2 / (2\sigma^2\kappa_\beta))}{\sum_k \exp(-\|\mu_i - \mu_k\|^2 / (2\sigma^2\kappa_\beta))}
\end{equation}

Let $d_{ij} = \|\mu_i - \mu_j\|$. For $d_{ij}$ significantly larger than the minimum distance, the exponential becomes negligible:
\begin{equation}
\exp(-d_{ij}^2 / (2\sigma^2\kappa_\beta)) \approx 0 \quad \text{when } d_{ij}^2 \gg 2\sigma^2\kappa_\beta \log N
\end{equation}

\textbf{Step 2: Effective threshold.}
Agents within the effective radius receive substantial attention:
\begin{equation}
\epsilon_{\text{eff}} = \sigma\sqrt{2\kappa_\beta \log N}
\end{equation}

Beyond this radius, attention decays exponentially to zero.

\textbf{Step 3: Approximate dynamics.}
The natural gradient flow becomes:
\begin{align}
\frac{d\mu_i}{dt} &\propto \sum_j \beta_{ij} (\mu_j - \mu_i) \\
&\approx \sum_{j : d_{ij} < \epsilon_{\text{eff}}} \frac{1}{|N_i(\epsilon)|} (\mu_j - \mu_i) \\
&= \text{avg}\{\mu_j - \mu_i : \|\mu_j - \mu_i\| < \epsilon_{\text{eff}}\}
\end{align}

which matches the Hegselmann-Krause update (in continuous time). \qed
\end{proof}

\subsection{Key Difference: Soft vs. Hard Threshold}

A crucial distinction separates the VFE framework from classical bounded confidence models. Hegselmann-Krause employs a hard cutoff, $\beta_{ij} = 1/|N_i|$ if $d < \epsilon$ and zero otherwise, creating a discontinuous attention function. The VFE framework instead produces a soft threshold via the exponential form $\beta_{ij} = \exp(-d^2 / (2\sigma^2\kappa_\beta)) / Z_i$, which decays smoothly with distance.

This distinction carries two significant implications. Regarding realism, actual social attention likely exhibits smooth falloff rather than abrupt cutoffs; people do not entirely ignore slightly-too-distant opinions but rather attend to them with diminishing weight. The soft threshold better captures this psychological reality. Regarding mathematical tractability, soft thresholds are differentiable everywhere, enabling analytic study of stability, bifurcations, and phase transitions that would be inaccessible with discontinuous dynamics.

\subsection{Adaptive Threshold}

Unlike fixed-$\epsilon$ models, the effective threshold in the VFE framework depends dynamically on multiple parameters through the relationship $\epsilon_{\text{eff}} = f(\sigma, \kappa_\beta, N)$. This functional dependence generates several novel predictions. The uncertainty effect predicts that higher epistemic uncertainty $\sigma$ increases tolerance for distant opinions (larger effective $\epsilon$), consistent with empirical findings that epistemic humility reduces polarization. The group size effect predicts that larger groups exhibit wider effective thresholds due to the $\log N$ scaling, potentially explaining why small online echo chambers often display more extreme views than larger populations. Finally, context-dependence implies that the same individual should exhibit different tolerance thresholds across different social contexts characterized by varying $\kappa_\beta$---a person might engage with distant opinions in academic settings (high $\kappa_\beta$) while ignoring them in partisan political contexts (low $\kappa_\beta$).

\subsection{Novel Predictions}

The bounded confidence derivation yields three testable predictions. Rather than sharp clustering boundaries, the soft threshold predicts gradual attention decay measurable via interaction frequencies across opinion distances. Experimental manipulation of $\kappa_\beta$---for instance, through platform design choices that encourage or discourage engagement with diverse content---should produce predictable changes in effective tolerance thresholds. Interventions that induce epistemic humility, operationalized as raising subjective uncertainty $\sigma$, should measurably increase cross-group engagement by widening the effective confidence bound.

\section{Derivation 5: Confirmation Bias (Requires Natural Gradient)}

\subsection{Phenomenon}

Confirmation bias---the tendency to update beliefs less from counter-attitudinal evidence than from supportive evidence---is one of the most robust findings in social psychology. It has been alternately explained as motivated reasoning, cognitive dissonance reduction, or Bayesian conservatism.

\subsection{Classical Approaches}

Most models treat confirmation bias as an \emph{ad hoc} deviation from rational Bayesian updating, requiring additional parameters (e.g., directional weighting of evidence). The VFE framework shows that asymmetric updating emerges naturally from the geometry of the statistical manifold.

\subsection{Derivation from VFE Framework}

\begin{proposition}[Confirmation Bias from Epistemic Inertia]
In the natural gradient flow regime, agents with high prior precision or many followers exhibit confirmation bias without any non-Bayesian mechanisms.
\end{proposition}

\begin{proof}[Proof Sketch]
\textbf{Step 1: Natural gradient with mass matrix.}
The update dynamics are:
\begin{equation}
\frac{d\mu_i}{dt} = -M_i^{-1} \nabla_{\mu_i} F
\end{equation}

where the full mass matrix is:
\begin{equation}
M_i = \Sigma_{p,i}^{-1} + \Sigma_{o,i}^{-1} + \sum_j \beta_{ij} \tilde{\Sigma}_{q,j}^{-1} + \sum_j \beta_{ji} \Sigma_{q,i}^{-1}
\end{equation}

with $\tilde{\Sigma}_{q,j}^{-1} = \Omega_{ij}\Sigma_{q,j}^{-1}\Omega_{ij}^T$ the transported neighbor precision.


\textbf{Step 2: Effect of high prior precision.}
When prior precision $\Sigma_p^{-1}$ is large (agent is confident in prior):
\begin{equation}
M_i^{-1} \approx (\Sigma_p^{-1})^{-1} = \Sigma_p \quad (\text{small})
\end{equation}

For the same force $\nabla F$, the velocity is small:
\begin{equation}
\frac{d\mu}{dt} = \Sigma_p \nabla F \ll \nabla F \quad \text{when } \Sigma_p \text{ is small}
\end{equation}

\textbf{Step 3: Effect of social mass.}
When $\sum_j \beta_{ji} \Sigma_{q,i}^{-1}$ is large (many followers paying attention to agent $i$):
\begin{equation}
M_i \text{ large} \implies M_i^{-1} \text{ small} \implies \frac{d\mu_i}{dt} \text{ small}
\end{equation}

Note the crucial distinction: $\beta_{ji}$ (others attending to $i$) contributes to $i$'s mass, not $\beta_{ij}$ (whom $i$ attends to).


\textbf{Step 4: Asymmetric response to evidence.}
Consider evidence $o$ that contradicts agent $i$'s belief. The observation term contributes:
\begin{equation}
\nabla_{\mu_i} F_{\text{obs}} \propto (\mu_i - \mu_{\text{evidence}})
\end{equation}

Update magnitude:
\begin{equation}
\Delta \mu_i \propto M_i^{-1} (\mu_i - \mu_{\text{evidence}}) = \left[\Sigma_p^{-1} + \Sigma_o^{-1} + \sum_j \beta_{ij} \tilde{\Sigma}_{q,j}^{-1} + \sum_j \beta_{ji} \Sigma_{q,i}^{-1}\right]^{-1} (\mu_i - \mu_{\text{evidence}})
\end{equation}

The magnitude of belief updates thus depends inversely on the total epistemic mass. Agents with high prior precision (low $\Sigma_p$) develop large mass and correspondingly small updates. Those with strong sensory grounding (low $\Sigma_o$) remain anchored against social pressure. Agents with many followers (large $\sum_j \beta_{ji}$) accumulate substantial outgoing social mass, rendering them resistant to new information. Even agents who merely attend to confident neighbors (large $\sum_j \beta_{ij}\tilde{\Sigma}_{q,j}^{-1}$) develop increased incoming social mass. Through these four mechanisms, confirmation bias emerges naturally: beliefs become resistant to contradictory evidence not through motivated reasoning but through the geometric structure of the statistical manifold. \qed
\end{proof}

\subsection{Critical Assumption: Natural Gradient}

This derivation requires interpreting the dynamics as \emph{natural gradient descent} (flow on the statistical manifold) rather than standard Euclidean gradient descent. The mass matrix $M^{-1}$ appears only in the natural gradient formulation.

\paragraph{Justification.} Natural gradient descent is the unique geometrically-principled learning rule on statistical manifolds, invariant under reparameterization (Amari, 1998). The Fisher-Rao metric $M$ is the Riemannian metric on the space of probability distributions. Standard gradient descent is only appropriate for Euclidean spaces; beliefs are distributions, not vectors.

\paragraph{Status.} This derivation is solid \emph{if} we accept natural gradient as the correct dynamics for learning on manifolds---a standard assumption in information geometry.

\subsection{Social Component of Confirmation Bias}

The novel insight of this derivation concerns the \emph{outgoing social mass} term $\sum_j \beta_{ji} \Sigma_{q,i}^{-1}$---representing the recoil from influencing others. This term generates three distinctive predictions. First, confirmation bias should be audience-dependent: the same individual should exhibit stronger resistance to counter-evidence when they have many followers (as with public intellectuals compared to private citizens), because $\sum_j \beta_{ji}$ increases with audience size. Second, the effect is self-precision-weighted, scaling with agent $i$'s own confidence $\Sigma_{q,i}^{-1}$; confident leaders therefore experience more rigidity than uncertain ones with identical follower counts. Third, the effect is dynamic: as attention structure changes---for instance, when content goes viral---confirmation bias should increase in real-time through the evolving $\beta_{ji}$ terms.

\subsection{Novel Predictions}

The confirmation bias derivation yields three experimentally tractable predictions. The follower effect predicts that manipulating perceived audience size should produce corresponding changes in resistance to counter-evidence. The precision effect, somewhat counterintuitively, predicts that counter-evidence from highly confident sources may be \emph{more} strongly resisted if it triggers defensive responses that increase the target's own precision. The temporal dynamics prediction suggests that sudden attention spikes---as when content goes viral---should produce temporarily elevated confirmation bias during the attention peak, with subsequent relaxation as attention disperses.

\section{Derivation 6: Social Impact Theory (Interpretive Mapping)}

\subsection{Classical Formulation}

Latané's Social Impact Theory (1981) posits that social influence is a multiplicative function of three factors:
\begin{equation}
\text{Impact} = f(\text{Strength} \times \text{Immediacy} \times \text{Number})
\label{eq:sit_classical}
\end{equation}

Here strength encompasses the expertise, status, and power of the influence source; immediacy captures proximity in space or time; and number reflects how many sources are simultaneously present. This qualitative principle has been widely applied across social psychology but has historically lacked a quantitative mechanistic formulation that would render it precisely testable.

\subsection{Mapping to VFE Framework}

The mass matrix \eqref{eq:mass_matrix} provides a natural quantitative interpretation of Latané's factors:
\begin{equation}
M_i = \Sigma_p^{-1} + \Sigma_o^{-1} + \sum_j \beta_{ij} \Omega_{ij} \Sigma_{q,j}^{-1} \Omega_{ij}^T + \sum_j \beta_{ji} \Sigma_{q,i}^{-1}
\end{equation}

\paragraph{Strength $\leftrightarrow$ $\Sigma_{q,j}^{-1}$.} Source precision---operationalized as confidence or expertise---contributes directly to the mass experienced by the target. Experts possess high precision (large $\Sigma_q^{-1}$) and consequently make large contributions to target mass, while uncertain sources with low precision contribute negligibly. This provides a mechanistic account of why expert opinions carry more weight: they literally increase the epistemic inertia of those who attend to them.

\paragraph{Immediacy $\leftrightarrow$ Transport penalty $\|\Omega_{ij} - I\|$.} Spatial and temporal distance enter the framework through gauge transformations between reference frames. Agents who are ``close'' in the relevant sense have nearly aligned frames ($\Omega_{ij} \approx I$), incurring small KL penalties and receiving high attention weights $\beta_{ij}$. Distant agents, whose frames are significantly rotated, incur large KL penalties and receive correspondingly low attention. The immediacy function thus emerges naturally from the softmax mechanism:
\begin{equation}
f_{\text{imm}}(\Omega_{ij}) = \exp(-\|\Omega_{ij} - I\|_F^2 / \kappa_\beta)
\end{equation}

\paragraph{Number $\leftrightarrow$ $\sum_j$.} More sources → more terms in sum → larger social mass.

\subsection{Quantitative Formulation}

The social mass contribution from source $j$ is:
\begin{equation}
\Delta M_i^{(j)} = \beta_{ij} \Omega_{ij} \Sigma_{q,j}^{-1} \Omega_{ij}^T
\end{equation}

In scalar form (for simplicity):
\begin{equation}
\Delta M_i^{(j)} \approx (\text{Number: }1) \times (\text{Strength: }\Sigma_{q,j}^{-1}) \times (\text{Immediacy: }f(\Omega_{ij}))
\end{equation}

Total impact:
\begin{equation}
M_i \approx \Sigma_p^{-1} + \Sigma_o^{-1} + \sum_j (\text{Strength}_j \times \text{Immediacy}_{ij} \times \beta_{ij}) + \sum_j (\text{Self-precision} \times \beta_{ji})
\label{eq:sit_vfe}
\end{equation}

where the final term captures ``recoil''---having followers increases one's own inertia.


\subsection{What the VFE Framework Adds}

\paragraph{Exact quantitative formula.} Latané's principle is qualitative; \eqref{eq:sit_vfe} gives precise predictions testable via measurement of $\Sigma_q$, $\Omega$, and $\beta$.

\paragraph{Time-varying impact.} As beliefs and attention evolve, mass changes dynamically:
\begin{equation}
M_i(t) = \Sigma_p^{-1} + \Sigma_o^{-1} + \sum_j \beta_{ij}(t) \tilde{\Sigma}_{q,j}(t)^{-1} + \sum_j \beta_{ji}(t) \Sigma_{q,i}(t)^{-1}
\end{equation}

This captures how influence waxes and wanes with changing confidence, attention, and audience structure.

\paragraph{Asymmetry.} Social impact is not reciprocal:
\begin{equation}
\Delta M_i^{(j)} \neq \Delta M_j^{(i)}
\end{equation}

High-status sources influence low-status targets more than the reverse, even with symmetric communication.

\paragraph{Testability.} Each component of the VFE formulation admits independent measurement: strength through confidence ratings and expertise assessments, immediacy through physical distance, temporal lags, and measures of frame alignment, and number through network degree and audience size metrics. The framework becomes testable by measuring these components separately and then verifying whether their combined effect matches the quantitative predictions of \eqref{eq:sit_vfe}.

\subsection{Caveat: Interpretive Mapping}

This is an \emph{interpretive correspondence}, not a formal equivalence. Latané's formula is intentionally vague (``multiplicative function of''); the VFE framework provides one specific quantitative instantiation. Other interpretations are possible, though \eqref{eq:sit_vfe} has the virtue of being uniquely derived from information geometry.

\subsection{Novel Predictions}

The Social Impact Theory mapping generates three distinctive predictions. The confidence modulation prediction states that impact should scale with source confidence $\Sigma_q^{-1}$, testable by experimentally manipulating perceived expertise while holding other factors constant. The frame alignment prediction addresses the phenomenon of ``talking past each other'': sources operating in different conceptual frameworks (high $\|\Omega - I\|$) should exert reduced impact even when physically proximate, suggesting that shared language and conceptual alignment matter independently of spatial proximity. The temporal decay prediction states that impact should decline with communication lag according to a specific functional form, with decay rate determined by the attention temperature $\kappa_\beta$.

\section{Summary Table: Derivation Quality}

\begin{table}[h]
\centering
\begin{tabular}{@{}lccp{6cm}@{}}
\toprule
\textbf{Model} & \textbf{Rigor} & \textbf{Status} & \textbf{Notes} \\
\midrule
DeGroot & $\checkmark\checkmark\checkmark$ & Rigorous & Exact limit with QED proof \\
Friedkin-Johnsen & $\checkmark\checkmark\checkmark$ & Rigorous & Exact limit; emergent stubbornness \\
Echo Chambers & $\checkmark\checkmark\checkmark$ & Rigorous & Direct consequence of softmax \\
Bounded Confidence & $\checkmark\checkmark$ & Solid & Soft approximation of hard threshold \\
Confirmation Bias & $\checkmark\checkmark$ & Solid & Requires natural gradient assumption \\
Social Impact Theory & $\checkmark\checkmark$ & Interpretive & Qualitative correspondence \\
\bottomrule
\end{tabular}
\caption{Quality assessment of each derivation. Rigor levels: $\checkmark\checkmark\checkmark$ = Rigorous (formal proof), $\checkmark\checkmark$ = Solid (strong approximation or additional assumptions), $\checkmark$ = Speculative (preliminary).}
\label{tab:rigor}
\end{table}

\section{Novel Predictions Summary}

Beyond recovering classical models, the VFE framework generates novel empirical predictions that distinguish it from existing theories.

The prediction of \emph{dynamic attention} states that influence networks should restructure as beliefs converge or diverge, with $\beta_{ij}(t)$ evolving in measurable ways via longitudinal network data. The \emph{epistemic inertia} prediction asserts that agents receiving high social attention should update their beliefs more slowly than peripheral agents, controlling for prior strength; this is testable via social media data (Twitter, Metaculus) tracking opinion leaders' update rates. The \emph{uncertainty dynamics} prediction states that subjective confidence should increase with consensus and decrease with exposure to diverse views, requiring measurement of confidence intervals rather than point estimates alone.

The \emph{phase transition} prediction draws a direct analogy to thermodynamics: polarization should emerge sharply at a critical attention selectivity $\kappa_\beta^{\text{crit}}$ rather than gradually, exhibiting the hallmarks of a genuine phase transition. The prediction of \emph{context-dependent stubbornness} contradicts trait-based theories by asserting that the same individual should exhibit variable resistance to influence across different contexts and network positions. The \emph{asymmetric influence} prediction states that impact should scale with source precision $\Sigma_q^{-1}$ independently of network centrality, implying that confident sources exert disproportionate influence even when structurally equivalent to uncertain ones.

Finally, the speculative prediction of \emph{underdamped oscillations} suggests that in low-friction environments characteristic of rapid social media discourse, beliefs may overshoot equilibrium and oscillate before settling, a phenomenon potentially measurable via high-frequency polling of opinion dynamics.

Testing even a subset of these predictions would provide strong empirical validation of the unifying framework, while falsification of any would constrain the parameter regimes where the theory applies.

\section{Conclusion}

We have demonstrated that six major social influence models---DeGroot social learning, Friedkin-Johnsen opinion dynamics, bounded confidence, confirmation bias, Social Impact Theory, and echo chambers---emerge as limiting cases or regimes of the Hamiltonian Variational Free Energy framework. Three derivations are rigorously proven (DeGroot, Friedkin-Johnsen, echo chambers), and three are solid approximations or interpretive mappings (bounded confidence, confirmation bias, Social Impact Theory).

This unification reveals that classical models differ not in fundamental mechanisms but in \emph{parameter regimes}: friction ($\gamma$), uncertainty ($\Sigma$), attention temperature ($\kappa_\beta$), and self-coupling ($\alpha$). The framework generates novel testable predictions beyond classical theories, particularly regarding dynamic attention, epistemic inertia from social position, and phase transitions in polarization.

The key sociological insight is that \emph{social structure creates epistemic consequences through information geometry}. Agents receiving high social attention develop inertial mass that resists belief revision---not through irrationality or motivated reasoning, but as a geometric necessity on the statistical manifold. This provides a mechanistic foundation for phenomena like authority rigidity, influencer confirmation bias, and the ``curse of expertise.''

\appendix

\section{Appendix: Technical Details}

\subsection{Fisher-Rao Metric and Natural Gradient}

For a family of distributions $p_\theta(x)$ parameterized by $\theta \in \R^d$, the Fisher-Rao metric is:
\begin{equation}
g_{ij}(\theta) = \mathbb{E}_{p_\theta}\left[\frac{\partial \log p_\theta}{\partial \theta_i} \frac{\partial \log p_\theta}{\partial \theta_j}\right]
\end{equation}

This metric is the unique (up to scaling) Riemannian metric invariant under sufficient statistics transformations.

For multivariate Gaussian $\N(\mu, \Sigma)$, the Fisher metric in the $\mu$-coordinates is $\Sigma^{-1}$, exactly the mass matrix bare term. The social coupling term extends this to multi-agent settings.

Natural gradient descent minimizes a function $F(\theta)$ along geodesics:
\begin{equation}
\frac{d\theta}{dt} = -g^{-1}(\theta) \nabla_\theta F
\end{equation}

Standard gradient descent ($g = I$) is only correct for Euclidean parameter spaces; belief distributions live on curved manifolds.

\subsection{Christoffel Symbols and Geodesic Corrections}

In the Hamiltonian regime, the momentum equation includes geodesic corrections:
\begin{equation}
\frac{d\pi_i}{dt} = -\nabla_i F - \Gamma_{ijk} \pi^j \pi^k
\end{equation}

where:
\begin{equation}
\Gamma_{ijk} = \frac{1}{2} g^{i\ell} \left(\frac{\partial g_{j\ell}}{\partial \theta_k} + \frac{\partial g_{k\ell}}{\partial \theta_j} - \frac{\partial g_{jk}}{\partial \theta_\ell}\right)
\end{equation}

These terms ensure trajectories follow geodesics on the manifold. For static metrics ($\partial g/\partial \theta = 0$), they vanish; for parameter-dependent mass matrices (our case), they create nonlinear coupling between position and momentum.

\subsection{Symplectic Integration}

Hamiltonian dynamics preserve phase space volume (Liouville's theorem) and approximately conserve energy. Standard integrators (Runge-Kutta, etc.) violate these properties, leading to artificial energy drift.

Symplectic integrators (Verlet, leapfrog, Ruth) preserve the symplectic 2-form:
\begin{equation}
\omega = \sum_i d\theta_i \wedge d\pi_i
\end{equation}

guaranteeing long-term stability and bounded energy drift. For our simulations, we use:
\begin{align}
\pi(t + \tfrac{\Delta t}{2}) &= \pi(t) + \tfrac{\Delta t}{2} f(\theta(t)) \\
\theta(t + \Delta t) &= \theta(t) + \Delta t \, g^{-1}(\theta) \pi(t + \tfrac{\Delta t}{2}) \\
\pi(t + \Delta t) &= \pi(t + \tfrac{\Delta t}{2}) + \tfrac{\Delta t}{2} f(\theta(t + \Delta t))
\end{align}

where $f = -\nabla F - \Gamma \pi \pi - \gamma \pi$.

\bibliographystyle{plain}
\bibliography{references}

\end{document}